% Options for packages loaded elsewhere
\PassOptionsToPackage{unicode}{hyperref}
\PassOptionsToPackage{hyphens}{url}
%
\documentclass[
]{book}
\usepackage{amsmath,amssymb}
\usepackage{lmodern}
\usepackage{iftex}
\ifPDFTeX
  \usepackage[T1]{fontenc}
  \usepackage[utf8]{inputenc}
  \usepackage{textcomp} % provide euro and other symbols
\else % if luatex or xetex
  \usepackage{unicode-math}
  \defaultfontfeatures{Scale=MatchLowercase}
  \defaultfontfeatures[\rmfamily]{Ligatures=TeX,Scale=1}
\fi
% Use upquote if available, for straight quotes in verbatim environments
\IfFileExists{upquote.sty}{\usepackage{upquote}}{}
\IfFileExists{microtype.sty}{% use microtype if available
  \usepackage[]{microtype}
  \UseMicrotypeSet[protrusion]{basicmath} % disable protrusion for tt fonts
}{}
\makeatletter
\@ifundefined{KOMAClassName}{% if non-KOMA class
  \IfFileExists{parskip.sty}{%
    \usepackage{parskip}
  }{% else
    \setlength{\parindent}{0pt}
    \setlength{\parskip}{6pt plus 2pt minus 1pt}}
}{% if KOMA class
  \KOMAoptions{parskip=half}}
\makeatother
\usepackage{xcolor}
\usepackage{color}
\usepackage{fancyvrb}
\newcommand{\VerbBar}{|}
\newcommand{\VERB}{\Verb[commandchars=\\\{\}]}
\DefineVerbatimEnvironment{Highlighting}{Verbatim}{commandchars=\\\{\}}
% Add ',fontsize=\small' for more characters per line
\usepackage{framed}
\definecolor{shadecolor}{RGB}{248,248,248}
\newenvironment{Shaded}{\begin{snugshade}}{\end{snugshade}}
\newcommand{\AlertTok}[1]{\textcolor[rgb]{0.94,0.16,0.16}{#1}}
\newcommand{\AnnotationTok}[1]{\textcolor[rgb]{0.56,0.35,0.01}{\textbf{\textit{#1}}}}
\newcommand{\AttributeTok}[1]{\textcolor[rgb]{0.77,0.63,0.00}{#1}}
\newcommand{\BaseNTok}[1]{\textcolor[rgb]{0.00,0.00,0.81}{#1}}
\newcommand{\BuiltInTok}[1]{#1}
\newcommand{\CharTok}[1]{\textcolor[rgb]{0.31,0.60,0.02}{#1}}
\newcommand{\CommentTok}[1]{\textcolor[rgb]{0.56,0.35,0.01}{\textit{#1}}}
\newcommand{\CommentVarTok}[1]{\textcolor[rgb]{0.56,0.35,0.01}{\textbf{\textit{#1}}}}
\newcommand{\ConstantTok}[1]{\textcolor[rgb]{0.00,0.00,0.00}{#1}}
\newcommand{\ControlFlowTok}[1]{\textcolor[rgb]{0.13,0.29,0.53}{\textbf{#1}}}
\newcommand{\DataTypeTok}[1]{\textcolor[rgb]{0.13,0.29,0.53}{#1}}
\newcommand{\DecValTok}[1]{\textcolor[rgb]{0.00,0.00,0.81}{#1}}
\newcommand{\DocumentationTok}[1]{\textcolor[rgb]{0.56,0.35,0.01}{\textbf{\textit{#1}}}}
\newcommand{\ErrorTok}[1]{\textcolor[rgb]{0.64,0.00,0.00}{\textbf{#1}}}
\newcommand{\ExtensionTok}[1]{#1}
\newcommand{\FloatTok}[1]{\textcolor[rgb]{0.00,0.00,0.81}{#1}}
\newcommand{\FunctionTok}[1]{\textcolor[rgb]{0.00,0.00,0.00}{#1}}
\newcommand{\ImportTok}[1]{#1}
\newcommand{\InformationTok}[1]{\textcolor[rgb]{0.56,0.35,0.01}{\textbf{\textit{#1}}}}
\newcommand{\KeywordTok}[1]{\textcolor[rgb]{0.13,0.29,0.53}{\textbf{#1}}}
\newcommand{\NormalTok}[1]{#1}
\newcommand{\OperatorTok}[1]{\textcolor[rgb]{0.81,0.36,0.00}{\textbf{#1}}}
\newcommand{\OtherTok}[1]{\textcolor[rgb]{0.56,0.35,0.01}{#1}}
\newcommand{\PreprocessorTok}[1]{\textcolor[rgb]{0.56,0.35,0.01}{\textit{#1}}}
\newcommand{\RegionMarkerTok}[1]{#1}
\newcommand{\SpecialCharTok}[1]{\textcolor[rgb]{0.00,0.00,0.00}{#1}}
\newcommand{\SpecialStringTok}[1]{\textcolor[rgb]{0.31,0.60,0.02}{#1}}
\newcommand{\StringTok}[1]{\textcolor[rgb]{0.31,0.60,0.02}{#1}}
\newcommand{\VariableTok}[1]{\textcolor[rgb]{0.00,0.00,0.00}{#1}}
\newcommand{\VerbatimStringTok}[1]{\textcolor[rgb]{0.31,0.60,0.02}{#1}}
\newcommand{\WarningTok}[1]{\textcolor[rgb]{0.56,0.35,0.01}{\textbf{\textit{#1}}}}
\usepackage{longtable,booktabs,array}
\usepackage{calc} % for calculating minipage widths
% Correct order of tables after \paragraph or \subparagraph
\usepackage{etoolbox}
\makeatletter
\patchcmd\longtable{\par}{\if@noskipsec\mbox{}\fi\par}{}{}
\makeatother
% Allow footnotes in longtable head/foot
\IfFileExists{footnotehyper.sty}{\usepackage{footnotehyper}}{\usepackage{footnote}}
\makesavenoteenv{longtable}
\usepackage{graphicx}
\makeatletter
\def\maxwidth{\ifdim\Gin@nat@width>\linewidth\linewidth\else\Gin@nat@width\fi}
\def\maxheight{\ifdim\Gin@nat@height>\textheight\textheight\else\Gin@nat@height\fi}
\makeatother
% Scale images if necessary, so that they will not overflow the page
% margins by default, and it is still possible to overwrite the defaults
% using explicit options in \includegraphics[width, height, ...]{}
\setkeys{Gin}{width=\maxwidth,height=\maxheight,keepaspectratio}
% Set default figure placement to htbp
\makeatletter
\def\fps@figure{htbp}
\makeatother
\setlength{\emergencystretch}{3em} % prevent overfull lines
\providecommand{\tightlist}{%
  \setlength{\itemsep}{0pt}\setlength{\parskip}{0pt}}
\setcounter{secnumdepth}{5}
\usepackage{booktabs}
\ifLuaTeX
  \usepackage{selnolig}  % disable illegal ligatures
\fi
\usepackage[]{natbib}
\bibliographystyle{plainnat}
\IfFileExists{bookmark.sty}{\usepackage{bookmark}}{\usepackage{hyperref}}
\IfFileExists{xurl.sty}{\usepackage{xurl}}{} % add URL line breaks if available
\urlstyle{same} % disable monospaced font for URLs
\hypersetup{
  pdftitle={Leaflet book},
  pdfauthor={Samuel Gachuhi Ngugi},
  hidelinks,
  pdfcreator={LaTeX via pandoc}}

\title{Leaflet book}
\author{Samuel Gachuhi Ngugi}
\date{2023-04-08}

\usepackage{amsthm}
\newtheorem{theorem}{Theorem}[chapter]
\newtheorem{lemma}{Lemma}[chapter]
\newtheorem{corollary}{Corollary}[chapter]
\newtheorem{proposition}{Proposition}[chapter]
\newtheorem{conjecture}{Conjecture}[chapter]
\theoremstyle{definition}
\newtheorem{definition}{Definition}[chapter]
\theoremstyle{definition}
\newtheorem{example}{Example}[chapter]
\theoremstyle{definition}
\newtheorem{exercise}{Exercise}[chapter]
\theoremstyle{definition}
\newtheorem{hypothesis}{Hypothesis}[chapter]
\theoremstyle{remark}
\newtheorem*{remark}{Remark}
\newtheorem*{solution}{Solution}
\begin{document}
\maketitle

{
\setcounter{tocdepth}{1}
\tableofcontents
}
\hypertarget{about}{%
\chapter*{About}\label{about}}
\addcontentsline{toc}{chapter}{About}

This is a \emph{sample} book written in \textbf{Markdown}. You can use anything that Pandoc's Markdown supports; for example, a math equation \(a^2 + b^2 = c^2\).

\hypertarget{usage}{%
\section*{Usage}\label{usage}}
\addcontentsline{toc}{section}{Usage}

Each \textbf{bookdown} chapter is an .Rmd file, and each .Rmd file can contain one (and only one) chapter. A chapter \emph{must} start with a first-level heading: \texttt{\#\ A\ good\ chapter}, and can contain one (and only one) first-level heading.

Use second-level and higher headings within chapters like: \texttt{\#\#\ A\ short\ section} or \texttt{\#\#\#\ An\ even\ shorter\ section}.

The \texttt{index.Rmd} file is required, and is also your first book chapter. It will be the homepage when you render the book.

\hypertarget{render-book}{%
\section*{Render book}\label{render-book}}
\addcontentsline{toc}{section}{Render book}

You can render the HTML version of this example book without changing anything:

\begin{enumerate}
\def\labelenumi{\arabic{enumi}.}
\item
  Find the \textbf{Build} pane in the RStudio IDE, and
\item
  Click on \textbf{Build Book}, then select your output format, or select ``All formats'' if you'd like to use multiple formats from the same book source files.
\end{enumerate}

Or build the book from the R console:

\begin{Shaded}
\begin{Highlighting}[]
\NormalTok{bookdown}\SpecialCharTok{::}\FunctionTok{render\_book}\NormalTok{()}
\end{Highlighting}
\end{Shaded}

To render this example to PDF as a \texttt{bookdown::pdf\_book}, you'll need to install XeLaTeX. You are recommended to install TinyTeX (which includes XeLaTeX): \url{https://yihui.org/tinytex/}.

\hypertarget{preview-book}{%
\section*{Preview book}\label{preview-book}}
\addcontentsline{toc}{section}{Preview book}

As you work, you may start a local server to live preview this HTML book. This preview will update as you edit the book when you save individual .Rmd files. You can start the server in a work session by using the RStudio add-in ``Preview book'', or from the R console:

\begin{Shaded}
\begin{Highlighting}[]
\NormalTok{bookdown}\SpecialCharTok{::}\FunctionTok{serve\_book}\NormalTok{()}
\end{Highlighting}
\end{Shaded}

\hypertarget{introduction}{%
\chapter{Introduction}\label{introduction}}

\hypertarget{what-is-leaflet}{%
\section{What is Leaflet?}\label{what-is-leaflet}}

Something to do with leaves? Not really.Leaflet, when barescrapped to its most basic definition, is simply an open source JavaScript library for interactive maps. It was developed in 2011 by Volodymyr Agafonkin, a Ukrainain with a mathematical background.

\hypertarget{how-does-it-work}{%
\section{How does it work?}\label{how-does-it-work}}

Leaflet can work if every line of code is inside a \texttt{html} document, so long as the code appears under the \texttt{\textless{}script\textgreater{}} tag. However, for a neat work, especially working with complex maps, it is recommended you separate the \texttt{html} file from its other components of \texttt{main.js} and \texttt{style.css} files.

``HTML we know, but what are \texttt{main.js} and \texttt{style.css} files, you may ask?

Well, beginning with \texttt{html}, which stands for \textbf{Hypertext Markup Language}, it is the language that is \href{https://www.tutorialspoint.com/html/index.htm}{used in creating webpages}. By talking of language, it is actually the standard. I am yet to come across any webpage that is made up of everything apart from HTML. If you want to have a view of what HTML looks like, just right click any webpage and click \emph{Inspect} in Google Chrome and Firefox. A toolbar will appear at the bottom or side of the webpage, depending on your settings.

\begin{Shaded}
\begin{Highlighting}[]
\NormalTok{knitr}\SpecialCharTok{::}\FunctionTok{include\_graphics}\NormalTok{(}\FunctionTok{rep}\NormalTok{(}\StringTok{"D:/gachuhi/my{-}leaflet/inspect.jpg"}\NormalTok{,))}
\end{Highlighting}
\end{Shaded}

\includegraphics[width=50pt]{../inspect}

Scroll over to the \textbf{Element} tab and you will have something that looks like this:

\begin{Shaded}
\begin{Highlighting}[]
\NormalTok{knitr}\SpecialCharTok{::}\FunctionTok{include\_graphics}\NormalTok{(}\FunctionTok{rep}\NormalTok{(}\StringTok{"D:/gachuhi/my{-}leaflet/elements.jpg"}\NormalTok{))}
\end{Highlighting}
\end{Shaded}

\includegraphics[width=26.67in]{../elements}

The part encircled in red is the \texttt{html} that makes up the webpage for the \href{https://sammigachuhi.github.io/pro-gmo-website/}{ProGMO website} in this case.

So, I am a GIS specialist, I want to learn how to make a html website so as to use leaflet. Whereas this document does not provide an indepth view of a html document, html websites are made up of elements known as \texttt{tags}. Tags, normally indicated by angle brackets (\textless\textgreater) are what introduce any form of content into a webpage, be it a paragraph (

), an image (), video () and even an entire section (

,

,

). With this basic introduction, let's create a basic html page.

To create a html element along with many other programming files, such as \texttt{.js} and \texttt{.css} which we shall see later, we use a text editor. A good example of a text editor is VS code and Pycharm. Check their websites on their installation methods for your personal computer. For this book, we shall be using VS code.

Here is a basic html webpage.

\begin{verbatim}
<!DOCTYPE html>
<html lang="en">
    <head>
        <title>A basic html webpage</title>
        <meta charset="utf-8">
        <link rel="stylesheet" href="style.css">
    </head>
    <body>
        <div id="division-1">
            <p>Hello, World!</p>
        </div>
        <script src="main.js">

        </script>

    </body>
</html>
\end{verbatim}

Let's go through the above tags one by one.

\begin{enumerate}
\def\labelenumi{\arabic{enumi}.}
\item
  \textless!DOCTYPE html\textgreater{} - It is an ``information'' to the browser about what document type to expect.
\item
  \begin{itemize}
  \tightlist
  \item
    It is the container for all other HTML elements (except for the \textless!DOCTYPE\textgreater{} tag). The \texttt{lang} attribute is used to assist web engines know which language the website uses.
  \end{itemize}
\item
  \begin{itemize}
  \tightlist
  \item
    It is not displayed on the webpage as other tags, but contains the metadata of the webpage.
  \end{itemize}
\item
  \begin{itemize}
  \tightlist
  \item
    Can you guess? You had it right. Defines the title of the document. In our case, if you open the webpage assuming you created it in VS Code, the webpage shall be titled \emph{A basic html webpage} at the tab of your web-browser.
  \end{itemize}
\item
  \begin{itemize}
  \tightlist
  \item
    This is one of the metadata hosted by the

    tag. The

    tag defines, rather than contains, as in the case of

    the metadata of the html webpage. In our case, we have used the attribute \texttt{charset="utf-8"} to specify the encoding for HTML5 documents which is \texttt{utf-8}.
  \end{itemize}
\item
  - Defines the relationship between a document and an external resource. It has various attributes but \texttt{rel} and \texttt{href} have been used. The former specifies the relationship between the current document and the linked document/resource. The \texttt{rel} here references the \texttt{styles.css} file as the style sheet for our html. That is, the styles for our html are found in the \texttt{styles.css} file. \texttt{href} on the other hand points the html document to the path of the stylesheet --the \texttt{styles.css} file.
\item
  \begin{itemize}
  \tightlist
  \item
    This is the crux of your webpage. If nothing is within the

    tags, your webpage will be as empty as a blank sheet of paper. This tag is the home for all the other contents of the webpage such as headings, paragraphs, images, tables etc.
  \end{itemize}
\item
  \begin{itemize}
  \tightlist
  \item
    This is a special element that lets you group similar sets of content together on a web page. You can use it as a generic container for associating similar content. In the above html script, we have included an \texttt{\textless{}id\textgreater{}} attribute that is in other words, a unique identifier for this section of the webpage. \texttt{\textless{}id\textgreater{}s} are useful if you want to customize the appearance of a certain part of the webpage. es behave in a similar way, but the difference between and is that id has to be unique, while es can be used more than once.
  \end{itemize}
\item
  \begin{itemize}
  \tightlist
  \item
    It is used to embed executable code or data. In most cases it refers to JavaScript, which enhances interactivity.
  \end{itemize}
\end{enumerate}

If you may have noticed above, most HTML tags end with . With a few exceptions such as , almost all HTML tags end this way.

\hypertarget{javascript}{%
\section{JavaScript}\label{javascript}}

JavaScript, shortened to \texttt{.js} is the language of the web. It enhances interactivity to HTML files which without it remain just static. Think of \texttt{.js} as the life of the party while HTML is just the setting. Without \texttt{.js} creating webmaps would not be possible since adding them to a html file using \texttt{\textless{}script\textgreater{}} is what brings in the interactive web features to an otherwise blank html.

\hypertarget{css-files}{%
\section{CSS files}\label{css-files}}

CSS stands for \emph{Cascading Style Sheet}. The CSS defines how your HTML is to appear, such as color and size of text, background color of the HTML as well as the structure of your HTML page.

CSS is quite a huge field despite being simple. However, the html elements of a webpage are accompanied by a curly bracket containing the specified properties and values.

\begin{itemize}
\item
  Properties: These are human-readable identifiers that indicate which stylistic features you want to modify. For example, font-size, width, background-color.
\item
  Values: Each property is assigned a value. This value indicates how to style the property.
\end{itemize}

Using the example of our ProGMO website, this is how we would specify the \texttt{\textless{}body\textgreater{}} element of our webpage.

\begin{verbatim}
body {
  font-family: "Open Sans", sans-serif;
  color: #444444;
}
\end{verbatim}

The body is known as the selector. However, selectors can be more specific, such as specifying the exact

that should be displayed in a particular way. Using our html file example, if there were other

s apart from the one above, we would specify our first one in a CSS document like so:

\begin{verbatim}
#division-1 {
  font-family: "Open Sans", sans-serif;
  color: #343a40;
}
\end{verbatim}

The values of that particular

could be changed to whatever you like, so long as they correspond to the right property. If it were a class, the particular class, assuming they were several, would be selected with the convention:

\begin{verbatim}
.class_name {
property: value
prperty2: value2}
\end{verbatim}

You can view the style of a particular HTML element using the styles tab found in the inspect console. It is shown in yellow bounds for a chrome webpage. Firefox should have a similar one.

\begin{Shaded}
\begin{Highlighting}[]
\NormalTok{knitr}\SpecialCharTok{::}\FunctionTok{include\_graphics}\NormalTok{(}\FunctionTok{rep}\NormalTok{(}\StringTok{"D:/gachuhi/my{-}leaflet/elements2.jpg"}\NormalTok{))}
\end{Highlighting}
\end{Shaded}

\includegraphics[width=26.67in]{../elements2}

The \href{https://developer.mozilla.org/en-US/docs/Learn/CSS/First_steps/How_CSS_is_structured}{MDN website} provides a lot of information on HTML and CSS.

\hypertarget{cross}{%
\chapter{Cross-references}\label{cross}}

Cross-references make it easier for your readers to find and link to elements in your book.

\hypertarget{chapters-and-sub-chapters}{%
\section{Chapters and sub-chapters}\label{chapters-and-sub-chapters}}

There are two steps to cross-reference any heading:

\begin{enumerate}
\def\labelenumi{\arabic{enumi}.}
\tightlist
\item
  Label the heading: \texttt{\#\ Hello\ world\ \{\#nice-label\}}.

  \begin{itemize}
  \tightlist
  \item
    Leave the label off if you like the automated heading generated based on your heading title: for example, \texttt{\#\ Hello\ world} = \texttt{\#\ Hello\ world\ \{\#hello-world\}}.
  \item
    To label an un-numbered heading, use: \texttt{\#\ Hello\ world\ \{-\#nice-label\}} or \texttt{\{\#\ Hello\ world\ .unnumbered\}}.
  \end{itemize}
\item
  Next, reference the labeled heading anywhere in the text using \texttt{\textbackslash{}@ref(nice-label)}; for example, please see Chapter \ref{cross}.

  \begin{itemize}
  \tightlist
  \item
    If you prefer text as the link instead of a numbered reference use: \protect\hyperlink{cross}{any text you want can go here}.
  \end{itemize}
\end{enumerate}

\hypertarget{captioned-figures-and-tables}{%
\section{Captioned figures and tables}\label{captioned-figures-and-tables}}

Figures and tables \emph{with captions} can also be cross-referenced from elsewhere in your book using \texttt{\textbackslash{}@ref(fig:chunk-label)} and \texttt{\textbackslash{}@ref(tab:chunk-label)}, respectively.

See Figure \ref{fig:nice-fig}.

\begin{Shaded}
\begin{Highlighting}[]
\FunctionTok{par}\NormalTok{(}\AttributeTok{mar =} \FunctionTok{c}\NormalTok{(}\DecValTok{4}\NormalTok{, }\DecValTok{4}\NormalTok{, .}\DecValTok{1}\NormalTok{, .}\DecValTok{1}\NormalTok{))}
\FunctionTok{plot}\NormalTok{(pressure, }\AttributeTok{type =} \StringTok{\textquotesingle{}b\textquotesingle{}}\NormalTok{, }\AttributeTok{pch =} \DecValTok{19}\NormalTok{)}
\end{Highlighting}
\end{Shaded}

\begin{figure}

{\centering \includegraphics[width=0.8\linewidth]{_main_files/figure-latex/nice-fig-1} 

}

\caption{Here is a nice figure!}\label{fig:nice-fig}
\end{figure}

Don't miss Table \ref{tab:nice-tab}.

\begin{Shaded}
\begin{Highlighting}[]
\NormalTok{knitr}\SpecialCharTok{::}\FunctionTok{kable}\NormalTok{(}
  \FunctionTok{head}\NormalTok{(pressure, }\DecValTok{10}\NormalTok{), }\AttributeTok{caption =} \StringTok{\textquotesingle{}Here is a nice table!\textquotesingle{}}\NormalTok{,}
  \AttributeTok{booktabs =} \ConstantTok{TRUE}
\NormalTok{)}
\end{Highlighting}
\end{Shaded}

\begin{table}

\caption{\label{tab:nice-tab}Here is a nice table!}
\centering
\begin{tabular}[t]{rr}
\toprule
temperature & pressure\\
\midrule
0 & 0.0002\\
20 & 0.0012\\
40 & 0.0060\\
60 & 0.0300\\
80 & 0.0900\\
\addlinespace
100 & 0.2700\\
120 & 0.7500\\
140 & 1.8500\\
160 & 4.2000\\
180 & 8.8000\\
\bottomrule
\end{tabular}
\end{table}

\hypertarget{parts}{%
\chapter{Parts}\label{parts}}

You can add parts to organize one or more book chapters together. Parts can be inserted at the top of an .Rmd file, before the first-level chapter heading in that same file.

Add a numbered part: \texttt{\#\ (PART)\ Act\ one\ \{-\}} (followed by \texttt{\#\ A\ chapter})

Add an unnumbered part: \texttt{\#\ (PART\textbackslash{}*)\ Act\ one\ \{-\}} (followed by \texttt{\#\ A\ chapter})

Add an appendix as a special kind of un-numbered part: \texttt{\#\ (APPENDIX)\ Other\ stuff\ \{-\}} (followed by \texttt{\#\ A\ chapter}). Chapters in an appendix are prepended with letters instead of numbers.

\hypertarget{footnotes-and-citations}{%
\chapter{Footnotes and citations}\label{footnotes-and-citations}}

\hypertarget{footnotes}{%
\section{Footnotes}\label{footnotes}}

Footnotes are put inside the square brackets after a caret \texttt{\^{}{[}{]}}. Like this one \footnote{This is a footnote.}.

\hypertarget{citations}{%
\section{Citations}\label{citations}}

Reference items in your bibliography file(s) using \texttt{@key}.

For example, we are using the \textbf{bookdown} package \citep{R-bookdown} (check out the last code chunk in index.Rmd to see how this citation key was added) in this sample book, which was built on top of R Markdown and \textbf{knitr} \citep{xie2015} (this citation was added manually in an external file book.bib).
Note that the \texttt{.bib} files need to be listed in the index.Rmd with the YAML \texttt{bibliography} key.

The RStudio Visual Markdown Editor can also make it easier to insert citations: \url{https://rstudio.github.io/visual-markdown-editing/\#/citations}

\hypertarget{blocks}{%
\chapter{Blocks}\label{blocks}}

\hypertarget{equations}{%
\section{Equations}\label{equations}}

Here is an equation.

\begin{equation} 
  f\left(k\right) = \binom{n}{k} p^k\left(1-p\right)^{n-k}
  \label{eq:binom}
\end{equation}

You may refer to using \texttt{\textbackslash{}@ref(eq:binom)}, like see Equation \eqref{eq:binom}.

\hypertarget{theorems-and-proofs}{%
\section{Theorems and proofs}\label{theorems-and-proofs}}

Labeled theorems can be referenced in text using \texttt{\textbackslash{}@ref(thm:tri)}, for example, check out this smart theorem \ref{thm:tri}.

\begin{theorem}
\protect\hypertarget{thm:tri}{}\label{thm:tri}For a right triangle, if \(c\) denotes the \emph{length} of the hypotenuse
and \(a\) and \(b\) denote the lengths of the \textbf{other} two sides, we have
\[a^2 + b^2 = c^2\]
\end{theorem}

Read more here \url{https://bookdown.org/yihui/bookdown/markdown-extensions-by-bookdown.html}.

\hypertarget{callout-blocks}{%
\section{Callout blocks}\label{callout-blocks}}

The R Markdown Cookbook provides more help on how to use custom blocks to design your own callouts: \url{https://bookdown.org/yihui/rmarkdown-cookbook/custom-blocks.html}

\hypertarget{sharing-your-book}{%
\chapter{Sharing your book}\label{sharing-your-book}}

\hypertarget{publishing}{%
\section{Publishing}\label{publishing}}

HTML books can be published online, see: \url{https://bookdown.org/yihui/bookdown/publishing.html}

\hypertarget{pages}{%
\section{404 pages}\label{pages}}

By default, users will be directed to a 404 page if they try to access a webpage that cannot be found. If you'd like to customize your 404 page instead of using the default, you may add either a \texttt{\_404.Rmd} or \texttt{\_404.md} file to your project root and use code and/or Markdown syntax.

\hypertarget{metadata-for-sharing}{%
\section{Metadata for sharing}\label{metadata-for-sharing}}

Bookdown HTML books will provide HTML metadata for social sharing on platforms like Twitter, Facebook, and LinkedIn, using information you provide in the \texttt{index.Rmd} YAML. To setup, set the \texttt{url} for your book and the path to your \texttt{cover-image} file. Your book's \texttt{title} and \texttt{description} are also used.

This \texttt{gitbook} uses the same social sharing data across all chapters in your book- all links shared will look the same.

Specify your book's source repository on GitHub using the \texttt{edit} key under the configuration options in the \texttt{\_output.yml} file, which allows users to suggest an edit by linking to a chapter's source file.

Read more about the features of this output format here:

\url{https://pkgs.rstudio.com/bookdown/reference/gitbook.html}

Or use:

\begin{Shaded}
\begin{Highlighting}[]
\NormalTok{?bookdown}\SpecialCharTok{::}\NormalTok{gitbook}
\end{Highlighting}
\end{Shaded}


  \bibliography{book.bib,packages.bib}

\end{document}
